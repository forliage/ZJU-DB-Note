\section{可串行性测试}
\subsection{可串行性测试}

考虑一组事务$T_1,T_2,...,T_n$的某个调度。

优先图——一种有向图,其顶点为事务(名称)。

如果两个事务发生冲突,且$T_i$更早访问了产生冲突的数据项,则我们从$T_i$向$T_j$绘制一条弧。

我们可以用被访问的项来标记这条弧。

\subsection{冲突串行性测试}

如果一个调度满足以下条件,则它是冲突串行化的,当且仅当其优先图是无环的。

存在时间复杂度为$n^2$的环检测算法,其中$n$是图中顶点的数量。(更好的算法时间复杂度为$n+e$)

如果优先图是无环的,那么可通过对该图进行拓扑排序来获得可串行化顺序。这是一种与图的偏序一致的线性顺序。

\subsection{并发控制与可串行性测试}

并发控制协议允许并发调度,但要确保这些调度是冲突/视图可串行化的,并且是可恢复且无级联回滚的。

并发控制协议通常不会在优先级创建时对其进行检查。相反,协议会施加一种规则来避免非可串行化调度。

不同的并发控制协议在允许的并发量和产生的开销之间提供了不同的权衡。

可串行性测试有助于我们理解并发控制协议为何是正确的。