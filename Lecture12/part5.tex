\section{可串行化}

\subsection{可串行化}

基本假设:每个事务都能保持数据库一致性。

因此,一组事务的串行执行能保持数据库一致性。

如果一个(可能是并发的)调度等同于一个串行调度,那么它就是可串行化的。不同形式的调度等价性引出以下概念:
\begin{enumerate}
    \item 冲突可串行化
    \item 视图可串行化
\end{enumerate}

\subsection{事务的简化视图}

我们忽略读写指令之外的操作。

我们假设事务可以在读写操作之间对本地缓冲区中的数据进行任意计算。

我们简化调度仅由读写指令组成。

\subsection{冲突指令}

事务$T_i$和$T_j$的指令$I_i$和$I_j$分别冲突,当且仅当存在某个项目$Q$被$I_i$和$I_j$同时访问,并且这些指令中至少有一个写入了$Q$。
\begin{enumerate}
    \item $I_i=read(Q),I_j=read(Q)$.$I_i$和$I_j$不冲突。
    \item $I_i=read(Q),I_j=write(Q)$.它们冲突。
    \item $I_i=write(Q),I_j=read(Q)$.它们冲突。
    \item $I_i=write(Q),I_j=write(Q)$。它们冲突。
\end{enumerate}

直观的说,$I_i$和$I_j$之间的冲突会在它们之间强制形成一种(逻辑上的)时间顺序。

如果$I_i$和$I_j$在一个调度中是连续的,并且它们不冲突,那么即使在调度中交换它们的顺序,其结果也不变。
若两个操作是冲突的,则两者执行次序不可交换。若两个操作不冲突,则可以交换次序。

\subsection{冲突可串行性}

如果一个调度$S$可以通过一系列非冲突指令的交换转换为调度$S'$,则称$S$和$S'$是冲突等价的。

如果一个调度$S$与一个串行调度冲突等价,则称该调度$S$是冲突可串行化的。

