\section{关系集}

\subsection{关系和关系集}

1.Relationship:
\begin{itemize}
    \item 是两个或多个实体之间的一种关联
    \item 例如在银行模型中,实体集customer(客户)和实体集loan(贷款)之间存在"借款人"关系:
\end{itemize}

2.Relationship set:
\begin{itemize}
    \item 是同一类关系的全体实例的集合
    \item 用于ER图中描绘实体集之间的联系
    \item 例如,定义了一个关系集borrower(customer-name, loan-number),它包含所有"哪个客户借了哪笔贷款"的元组
\end{itemize}

\subsection{形式化定义}

假设有$n$个实体集$E_1,E_2,...,E_n$,则一个关系集可以形式地看作:
$$R\subset E_1\times E_2\times ...\times E_n$$
其中$(e_1,e_2,...,e_n)\in R$表示实体$e_i\in E_i(i=1...n)$之间存在一次关联(一个关系实例)

\subsection{关系集的属性}

关系属性就像实体的属性一样,用来描述关系本身的特征。

例如\texttt{depositor(customer-name,account-number,access-date)}这个关系集:
\begin{itemize}
    \item 它把\texttt{customer}与\texttt{account}关联起来,表示"哪个客户在什么时候访问了哪个账户"
    \item 其中\texttt{access-date}就是描述这次访问的一个属性
\end{itemize}

\subsection{关系集的度}

Degree表示一个关系集中参与的实体集数量(即上面定义里$n$的值)

最常见的binary relationship:degree=2,两个实体集之间的联系。

也可以有三元(ternary)或更高元(n-ary)的关系,但在实际建模中较少见。

\subsection{映射基数}

对于二元关系集,通常用映射基数来描述"一个实体最多能和多少个对方实体关联":
\begin{enumerate}
    \item 一对一:一个$A$中的实体最多关联一个$B$中的实体,反之亦然。例如:\texttt{country-president}
    \item 一对多:一个$A$中的实体可以关联多个$B$中的实体,但每个$B$中的实体仅对应一个$A$。例如:\texttt{department-employee}
    \item 多对一:对应上面一对多的反向视图:多个$A$对应同一个$B$,且每个$B$最多对应一个$A$。例如:\texttt{patient-doctor}
    \item 多对多:一个$A$可以与多个$B$关联,同时一个$B$也可与多个$A$关联。例如:\texttt{student-course}
\end{enumerate}

注意:在E-R图中,通常用在连线上标注 “1”, “N” 或 “M” 来表示各端的映射基数。

\subsection{关系集到关系模式的映射}

在将 ER 模型转换为关系模式时,每个二元关系集可以根据其映射基数不同,选择不同的实现方式:

\begin{enumerate}
    \item 一对一:可将二者合并为一张表;或在任意一侧添加对方主键作为外键。
    \item 一对多:在“多”端的关系表中,添加“1”端的主键作为外键。例如\texttt{employee}表中加\texttt{department\_id}列,引用\texttt{department(dept\_id)}
    \item 多对多:需要拆出一个关联表(交叉表),其主键是两端实体的主键组合。
\end{enumerate}