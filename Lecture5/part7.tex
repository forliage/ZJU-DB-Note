\section{实体-关系模式数据库设计}

\subsection{E-R 数据库模式设计流程}

(1)需求分析(Requirement analysis)
\begin{itemize}
    \item 目的:明确系统需要处理哪些数据、支持哪些应用程序、完成哪些操作。
    \item 产出:系统需求文档,作为后续设计的依据。
\end{itemize}

(2)概念数据库设计(Conceptual database design)
\begin{itemize}
    \item 使用 E-R 模型(或类似的高层数据模型)描述数据和约束。
    \item 目的是提供一个高层次的抽象描述,让业务人员、开发人员都能理解数据库结构。
\end{itemize}

(3)逻辑数据库设计(Logical database design)
\begin{itemize}
    \item 将概念设计转化为逻辑数据库模式,即表(tables)。
    \item 包括模式优化:关系规范化(Normalization),消除冗余和异常,确保数据一致性和高效性。
\end{itemize}

(4)物理数据库设计(Physical database design)
\begin{itemize}
    \item 涉及索引(Indexing)、聚簇(clustering)、数据库调优(database tuning)。
    \item 目标是优化数据库的性能,提升查询和事务处理速度。
\end{itemize}

\subsection{E-R 设计决策}

(1)用属性(attribute)还是实体集(entity set)表示对象?

决策依据:
\begin{itemize}
    \item 如果对象只有简单的名字或单值特性 ,可用属性。
    \item 如果对象本身有多个属性需要描述 , 应定义为实体集。
\end{itemize}

电话(phone)可以作为属性存储在 employee 中,但如果需要记录多个电话和电话的其他信息(位置、类型、颜色),则应定义 phone 为独立实体集。

设计原则:
\begin{itemize}
    \item 一个对象不能同时作为实体和属性。
    \item 实体集之间才能建立联系,属性不能单独与实体集建立联系。
\end{itemize}

(2)用实体集(entity set)还是关系集(relationship set)表示?

决策依据:
\begin{itemize}
    \item 如果要描述两个实体之间发生的“动作” , 用关系集。
    \item 如果描述的是对象本身 , 用实体集。
\end{itemize}

“借款人(borrower)” 是客户和贷款之间的关系,建模为关系集。“贷款(loan)” 是一个独立实体,建模为实体集。

设计影响因素:
\begin{itemize}
    \item 关系的映射基数(mapping cardinality)会影响建模方式。
    \item 关系集的设计有助于更清晰表达实体之间的交互行为。
\end{itemize}

(3)用实体的属性(attribute of an entity)还是关系(relationship)表示?

决策依据:需要从对象的独立性和减少数据冗余角度权衡。

在 student 表中直接存 supervisor-id、supervisor-name 等 supervisor 信息,会导致冗余。
更合理的设计是将 supervisor 定义为单独实体集,student 和 supervisor 之间通过关系(如 stu-sup)联系。

对于语义上独立的对象,应建模为实体集。关系集能更灵活表达动态变化的关联关系。

\subsection{其它 E-R 设计决策}

(4)使用三元或 n 元关系(ternary or n-ary relationship)还是一对二元关系(binary relationships)

如果三个或多个实体之间存在统一的语义关系,n元关系;若可拆解成多个二元关系且含义明确 ,二元关系

(5)使用强实体集(strong entity set)还是弱实体集(weak entity set)

如果实体有自己独立的主键,强实体集。如果实体必须依赖于另一个实体,且没有完全主键,弱实体集。

(6)使用特化/泛化(specialization/generalization)

有助于模块化设计,提高模型的灵活性和扩展性;通过特化/泛化机制,可以清晰表达实体之间的继承关系

(7)使用聚集(aggregation)

将 E-R 图的一部分组合成一个单独实体集,作为一个单元处理;便于处理关系之间的关系,提高设计的抽象层次