\section{键}

在E-R Diagram和关系模型中,Key是保证Entity或Relation中每个实例唯一可区分的属性或属性组合。

\subsection{实体集的键}

1.Super Key:
\begin{itemize}
    \item 定义:实体集$E$中的属性集$S\subset$所有属性,若对同一个实体集中的任何两个不同实体$e_1,e_2$,它们在$S$上的取值总不相同,则称$S$是$E$的一个super key.
    \item Super key可以包含荣誉属性;只要能够唯一标识实体,就算超键。
\end{itemize}

2.Candidate Key:
\begin{itemize}
    \item 定义:在所有超键中,如果一个超键$K$去掉任一属性后就不再能唯一标识实体(即再不是超键),则称$K$为一个候选键,也叫极小超键(minimal superkey)。
    \item 一个实体集可有多个候选键,例如学生实体集可能有 \{学号\} 和 \{身份证号\} 都是候选键。
    \item 在候选键中选择一个最具业务意义的,作为该实体集的主键(Primary Key)。
\end{itemize}

3.Primaey Key:
\begin{itemize}
    \item 从所有候选键中选出的、用来唯一标识实体的那个,在关系模型里常被声明为 PRIMARY KEY。
    \item 主键有以下特点:唯一性,非空性
\end{itemize}

\subsection{关系集的键}

对于一个二元关系集$R\subset E_1\times E_2$,参与实体集的主键组合本身就构成它的超键:

1.Super Key:将参与关系集的各实体集中对应的主键组合在一起,就能唯一标识一次关联。

2.Candidate Key:在大多数多对多(M:N)关系中,这个组合就是唯一的候选键;对于一对多(1:N)或一对一(1:1)的关系,有时可以只选组合中的一部分作为候选键(但要符合最小性和非空性)。选择哪个组合或子集作为主键,需要根据映射基数(Mapping Cardinality)及业务语义来决定,例如“作为外键的属性不应常变”、“关系属性不能为 NULL” 等。

\subsection{映射基数对键的影响}

\begin{itemize}
    \item 一对一:关系集本身可以合并到任意一侧实体表,也可用一侧实体的主键做关系表的主键/外键。
    \item 一对多:多端实体集的主键与对应一端主键组合的超键可简化为“多端的主键 + 一端外键”形式;若一端的外键属性在多端表中一定非空,则这个外键本身也能唯一标识关联,可能成为候选键。
    \item 多对多:通常需在中间关系表中用“双方主键组合”做主键;这是最小且唯一的候选键。
\end{itemize}