\section{弱实体集}

在E-R模型中,Weak Enitity Set是一类自身没有完全主键、其存在依赖于另一个"强实体集"的实体集合。

\subsection{什么是弱实体集}

弱实体集是指"没有自己固有的主键"的实体集:它的每个实体不能单凭自身属性唯一标识,必须借助与某个强实体
的关联才能区分。

弱实体集通常用双实线的矩形表示,关联它的标识性联系用双菱形表示。

\subsection{弱实体集的要素}

1.标识实体集:弱实体集依赖的强实体集,后者必须有一个完整的主键。

2.标识关系:把弱实体集和标识实体集连接起来的全参与、一对多联系:
\begin{itemize}
    \item 全参与:弱实体必须对应某个强实体才存在
    \item 一对多:一个强实体可关联多个弱实体,但每个弱实体仅对应一个强实体
\end{itemize}

3.部分码/辨别符:
\begin{itemize}
    \item 在弱实体集中,用来区分同一标识实体下不同实体的"局部唯一属性"
\end{itemize}

4.弱实体集的主键:由标识实体集的主键加上自身的部分码共同组成。

\subsection{转换到关系模式}

在关系模式中,弱实体集\texttt{section}和\texttt{payment}会映射为表,但不再单独存储”标识实体的主键“,因为
它隐含在外键列中:
\begin{lstlisting}[style=sqlstyle]
    CREATE TABLE course (
      course_id   INT      PRIMARY KEY,
      title       VARCHAR(50),
      credits     INT
    );
    
    CREATE TABLE section (
      sec_id      INT         NOT NULL,
      semester    CHAR(6)     NOT NULL,
      year        INT         NOT NULL,
      course_id   INT         NOT NULL, 
      PRIMARY KEY (course_id, sec_id, semester, year),
      FOREIGN KEY (course_id) REFERENCES course(course_id)
    );
    
    CREATE TABLE loan (
      loan_number VARCHAR(20) PRIMARY KEY,
      amount      DECIMAL(10,2)
    );
    
    CREATE TABLE payment (
      payment_number INT      NOT NULL,  
      payment_date   DATE     NOT NULL,
      payment_amount DECIMAL(10,2),
      loan_number    VARCHAR(20) NOT NULL,  
      PRIMARY KEY (loan_number, payment_number),
      FOREIGN KEY (loan_number) REFERENCES loan(loan_number)
    );
        
\end{lstlisting}

section 表中没有再声明单独的 course\_id 属性为主键,因为 course\_id 已包含在复合主键里。

外键 (course\_id) 与标识关系相对应,确保“无主课程的节次不能存在”。