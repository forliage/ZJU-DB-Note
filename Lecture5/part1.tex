\section{实体集}

在数据库设计中,Entity Set是从现实世界抽象出的一类"可区分对象"的集合,是构建ER模型的基本单元。

\subsection{实体和实体集}

1.Entity:
\begin{itemize}
    \item Entity是现实世界中可被唯一识别的"物"或"事",可以是具体的也可以是抽象的
    \item 每个实体由一组attributes来描述,例如student entity可能有id,name,age,gender,profession等属性
\end{itemize}

2.Entity Set:
\begin{itemize}
    \item Entity Set是一类具有相同属性集合的entity的集合,即同一类型实体的全体。
    \item 例如"Student"实体集包含所有学生;"Customer"实体集包含所有客户;"Loan"实体集包含所有贷款
    \item 在关系模型中,一个实体集最终映射为一张关系(表),实体映射为关系中的一行(元组),属性映射为列(字段)
\end{itemize}

\subsection{属性与域}

1.属性:
\begin{itemize}
    \item 属性是对实体某种特征或性质的描述。例如客户实体的属性包含customer-id,customer-name,customer-street,customer-city等
    \item 实体集customer对应关系的表头(列名)就是这些属性名称
\end{itemize}

2.域(Domain):
\begin{itemize}
    \item 域也称"值域"或"取值集合",指某个属性所允许的所有可能取值的集合
    \item 例如属性customer-id的域可以是所有符合特定编码规则的字符串;属性age的域可以是整数[0,150]
\end{itemize}

\subsection{属性类型}

1.简单属性 vs.复合属性
\begin{itemize}
    \item 简单属性:原子不可分。如sex,age
    \item 复合属性:有多个分量属性组成,如name可以分解为first-name,middle-initial,last-name;address可分解为street,city,state,postal-code等
\end{itemize}

2.单值属性 vs. 多值属性
\begin{itemize}
    \item 单值属性:对每个实体只能取一个值,例如某学生只有一个学号
    \item 多值属性:对某个实体可能取多个值,例如phone-numbers,一个客户有多个电话号码。在关系模型中多值属性要拆成单独的关联表来实现。
\end{itemize}

3.派生属性 vs. 基属性
\begin{itemize}
    \item 基属性:在数据库中物理存储的属性
    \item 派生属性:可以由其它属性计算或推导出来,不必实际存储,例如age可以由birthday和当前日期计算得出。
\end{itemize}

\subsection{实体集和关系模型}

在从ER模型向关系模型转换时,每个实体集会成为一张关系表;

属性映射为表的列;

实体映射为表中的一条记录(元组);

对于多值属性或复合属性,需要额外设计关联表或将复合属性展开为多个列