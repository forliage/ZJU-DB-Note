\section{扩展实体-关系特征}

\subsection{概述}

扩展E-R特征是在传统E-R模型基础上,增强了对复杂数据结构的建模能力,主要包括以下几个特征:
\begin{itemize}
    \item 分层实体集:特化、泛化、设计约束
    \item 聚集
\end{itemize}

\subsection{实体集的分层}

\subsubsection{特化、具体化}

含义:一种自顶向下的设计过程。

目的:将一个实体集中区分出子类,这些子类具有区别于其它实体的特性。

过程:
\begin{itemize}
    \item 从一个高层次的实体集出发,将具有特殊属性或关系的实体划分到子集
    \item 子集可以拥有额外的属性或参与额外的关系,而这些在父实体集中不存在
\end{itemize}

属性继承:低层次的实体集会继承高层次实体集的所有属性和关系参与情况。

\subsubsection{泛化、普遍化}

含义:一种自底向上的设计过程

目的:将多个具有共同特征的实体集合并成一个高层次的实体集

过程:
\begin{itemize}
    \item 从多个低层次的实体集出发,找出它们的共性,构成一个高层次的实体集
    \item 泛化和特化是逆过程,在E-R图中表示方法相同
\end{itemize}

注意:Specialization和Generalization可以互换使用,视设计场景而定。

\subsubsection{特化/泛化设计约束}

设计约束可以细化建模的规则,主要包括:

(1)基于哪些实体可以成为子集成员的约束

条件定义的:通过明确条件来决定实体是否属于某个子集。

用户定义的:根据业务需求人工定义,不依赖属性值。

(2)子集之间的互斥性约束

不相交:一个实体只能属于一个低层次实体集。E-R图中标记为disjoin或Disj

可重叠:一个实体可以同时属于多个低层次实体集。

(3)完全性约束:指定高层实体集中是否必须出现在至少一个低层实体集中。

完全泛化:高层实体集中的实体必须属于至少一个低层实体集。E-R 图中 ISA 使用双线表示。

部分泛化:高层实体集中的实体可以不属于任何低层实体集。E-R 图中 ISA 使用单线表示。

\subsection{聚集}

问题:如何表达关系之间的关系?有些场景中,关系本身还需要参与到另一个关系中。

解决方案 → 聚集(Aggregation):
\begin{itemize}
    \item 将关系(如 works-on)视为一个抽象实体,可以再与其他实体形成关系
    \item 这样可以避免冗余信息,并清晰表达“关系之间的关系”。
\end{itemize}

优势:
\begin{itemize}
    \item 支持关系之间的关系建模。
    \item 允许将关系抽象为新实体进行管理。
\end{itemize}



