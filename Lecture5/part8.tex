\section{将实体-关系模式转换为表}

\subsection{概述}

基本原则:
\begin{itemize}
    \item 将 E-R 图 转换为表,是实现 关系数据库设计的关键步骤。
    \item 一个符合 E-R 图结构的数据库 → 可以用 一组表(collection of tables) 来表示。
    \item 每一个实体集(entity set)和关系集(relationship set) 都对应一个唯一的表,表名通常与实体集或关系集同名。
\end{itemize}

\subsection{表示实体集为表}

强实体集(Strong Entity Set):
\begin{itemize}
    \item 直接转换为表,表的属性 = 实体的属性。
    \item 主键直接作为表的主键。
\end{itemize}

\subsection{复合属性}

转换规则:
\begin{itemize}
    \item 将复合属性拆分成 多个简单属性。
    \item 每个组成部分作为单独的属性出现在表中。
\end{itemize}

原则:提取所有“叶子节点”属性放入表中。

\subsection{多值属性}

转换规则:
\begin{itemize}
    \item 为每一个多值属性创建单独的表。
    \item 新表的主键 = 原实体的主键 + 多值属性本身。
\end{itemize}

\subsection{弱实体集}

转换规则:
\begin{itemize}
    \item 弱实体集自身不能独立存在,依赖于强实体集的主键。
    \item 转换成表时需要包含:自身的属性;强实体集的主键(作为外键)
\end{itemize}

\subsection{关系集的表示}

转换规则:关系集转为表时,表中应包含:参与关系的实体集的主键(作为外键)。关系集自身的属性(如果有).

多对一 or 一对多:可以选择在“多”端表中添加一个额外属性,存储“1”端实体的主键。

\subsection{表的冗余}

Total Participation(完全参与):直接将 "1" 端的主键加到 "多" 端表中作为属性。

Partial Participation(部分参与):若将主键加到 "多" 端表中,可能导致 null 值 出现。此时是否拆出单独表需根据场景权衡。

关系集与弱实体的冗余:连接弱实体与其标识强实体的关系集 → 通常是冗余的,因为弱实体表中已经包含了强实体的主键。

\subsection{特化/泛化的表示}

方法1:为高层实体集建一个表;为每个低层实体集单独建表,包含高层主键 + 低层特有属性。访问完整信息需联表查询。



方法2:为每个实体集建立表,包含本地属性 + 继承的属性。若特化是 total,可省略高层表(用 view 替代)。可能出现冗余,尤其是 street、city 出现在多个表中,数据一致性需维护。