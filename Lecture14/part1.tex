\section{故障分类}

\subsection{故障分类}

事务故障:
\begin{itemize}
    \item 逻辑错误:由于某些内部错误,事务无法完成。条件:溢出、输入错误、未找到数据等
    \item 系统错误:由于错误条件(例如死锁),数据库系统必须终止活动事务。错误条件(例如死锁)
\end{itemize}

系统崩溃:停电或其它硬件或软件故障导致系统崩溃。

故障停止假设:假定非易失性存储内容不会因系统崩溃而损坏(数据库系统有大量完整性检查,以防止磁盘数据损坏)

磁盘故障:磁头损坏或类似的磁盘故障会破坏全部或部分磁盘存储的:假定损坏是可检测的:磁盘驱动器使用校验和检测故障。

\subsection{恢复算法}

考虑事务$T_i$,它将\$50从账户A转移到账户B。两次更新:从账户A中减去50并向账户B中添加50.

事务$T_i$需要对A和B的更新输出到数据库。
\begin{itemize}
    \item 在进行了其中一项修改单两项修改都未完成之前,可能会发生故障。
    \item 在未确保事务会提交的情况下修改数据库可能会使数据库处于不一致状态
    \item 如果在事务提交后立即发生故障,不修改数据库可能会导致更新丢失
\end{itemize}

恢复算法分为两部分:
\begin{enumerate}
    \item 在正常事务处理期间采取的操作,以确保有足够的信息来从故障中恢复
    \item 故障发生后采取的操作,将数据库内容恢复到确保原子性、一致性和持久性的状态
\end{enumerate}