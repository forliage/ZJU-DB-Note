\section{并发事务恢复}

我们修改基于日志的恢复方案,以允许多个事务并发执行
\begin{itemize}
    \item 所有事务共享一个磁盘缓冲区和一个日志文件
    \item 一个缓冲区可以包含由一个或多个事务更新的数据项
\end{itemize}

我们假设使用严格两阶段锁进行并发控制(2PL,X锁保持到事务结束)。即未提交事务的更新对其它事务不可见。

日志记录如前所述进行,不同事务的日志记录在日志中穿插排列。

检查点技术和恢复时采取的操作必须改变。因为在执行检查点时可能有多个事务处于活动状态。

检查点的执行方式与之前相同,只是现在检查点日志记录的形式为<checkpoint $L$>,其中$L$是检查点执行时活跃事务的列表。

我们假设在执行检查点时没有更新操作正在进行。

当系统从奔溃中恢复中:

\noindent 1.它首先执行以下操作:
\begin{itemize}
    \item 将撤销列表和重做列表初始化为空
    \item 从日志末尾开始反向扫描,当找到第一条<checkpoint $L$>记录时停止扫描。同时,反向扫描期间,对于日志中找到的每条记录:
       \begin{enumerate}
          \item 如果记录是<$T_i$ commit>,则将$T_i$添加到重做列表中
          \item 如果记录是<$T_i$ start>,那么如果$T_i$不在重做列表中,则将$T_i$添加到撤销列表
          \item 如果记录是<$T_i$ abort>,则将$T_i$添加到撤销列表
       \end{enumerate}
    \item 对于$L$中的每个$T_i$,如果$T_i$不在重做列表中,则将$T_i$添加到撤销列表。此时撤销列表包含必须撤销的未完成事务,重做列表包含必须重做的已完成事务。
\end{itemize}

\noindent 2.现在恢复过程如下:
\begin{itemize}
    \item 从日志文件末尾开始方向扫描日志,当为撤销列表中的每个$T_i$都遇到<$T_i$ start>记录时停止。在扫描期间,对属于撤销列表中事务的每个日志 记录执行撤销操作。
    \item 定位最近的<checkpoint $L$>记录 
    \item 从<checkpoint $L$>记录开始向前扫描日志,直到日志末尾。在扫描期间,对属于重做列表中某个事务的每个日志记录执行重做操作。
\end{itemize}