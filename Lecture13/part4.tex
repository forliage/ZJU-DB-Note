\section{插入和删除操作}

使用两阶段锁:
\begin{itemize}
    \item 仅当删除元组的事务对要删除的元组具有排他锁时,才可以执行删除操作。
    \item 项数据库中插入新元组的事务会获得该元组的X模式锁
\end{itemize}

插入和删除操作可能会导致幻影现象。

\noindent 扫描关系的事务(例如,查找佩里里奇所有账户余额的总和)以及在关系中插入一个元组的事务
(例如,在佩里里奇插入一个新账户)(从概念上讲)进官没有共同访问任何元组,但仍会发生冲突。

如果仅使用元组锁,则可能会产生不可串行化的调度。例如,扫描事务看不到新账户,
但会读取更新事务写入的其它一些元组。

扫描该关系的事务正在读取指示该关系包含哪些元组的信息,而插入元组的事务会更新相同的信息。
该信息应被加锁。

一种解决方案:将一个数据项与该关系关联起来,以表示关于该关系包含哪些元组的信息;
扫描该关系的事务在该数据项上获取共享锁。插入或删除元组的事务在该数据项上获取排他锁(注意:数据项上的锁与
单个元组上的锁不冲突)

上述协议在插入/删除操作方面提供的并发度非常低。

索引锁定协议通过对某些索引桶加锁,在防止幻象现象的同时提供了更高的并发度。

\noindent\textbf{索引锁定协议}

每个关系必须至少一个索引。对关系的访问必须仅通过该关系的索引之一进行。

执行查找操作的事务$T_i$必须以共享(S)模式锁定其访问的所有索引桶。

事务$T_i$在未更新关系$r$的所有索引情况下,不得将元组$t_i$插入到关系$r$中。

$T_i$必须对每个索引执行查找操作,以找到所有可能包含指向元组$t_i$的指针的索引桶(假设该元组已经存在),
并以X模式锁定这些索引桶。$T_i$还必须以X模式锁定其修改的所有索引桶。

必须遵守两阶段锁定协议的规则。

