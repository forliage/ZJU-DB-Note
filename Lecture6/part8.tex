\section{第四范式}

\subsection{第四范式(4NF)定义}

一个关系模式 $R$ 关于一组函数依赖和多值依赖 $D$,若对于 $D^+$ 中所有形如 $\alpha \twoheadrightarrow \beta$ 的多值依赖,$\alpha \subseteq R, \beta \subseteq R$,满足以下至少一个条件,则称 $R$ 满足第四范式(4NF):

\begin{itemize}
    \item $\alpha \twoheadrightarrow \beta$ 是平凡的(trivial):
    \begin{itemize}
        \item $\beta \subseteq \alpha$ 或
        \item $\alpha \cup \beta = R$
    \end{itemize}
    \item $\alpha$ 是关系 $R$ 的一个超码(superkey)。
\end{itemize}

\textbf{性质}:

\begin{itemize}
    \item 如果一个关系在 4NF 中,它一定在 BCNF 中。
\end{itemize}

\subsection{为什么需要4NF?}

\begin{itemize}
    \item 有些在 BCNF 中的关系模式仍存在\textbf{多值依赖}(Multivalued Dependencies, MVDs)导致的冗余和异常。
    \item 多值依赖的典型特征是某些属性组之间的值是独立的。
\end{itemize}

\textbf{例子}:

\begin{itemize}
    \item 关系 \texttt{classes(course, teacher, book)},对于每门课程 \texttt{course},有一组 \texttt{teacher} 和一组 \texttt{book},且两者相互独立:
    \begin{itemize}
        \item $course \twoheadrightarrow teacher$
        \item $course \twoheadrightarrow book$
    \end{itemize}
\end{itemize}

\textbf{问题}:

\begin{itemize}
    \item 数据冗余(Redundant):
    \begin{itemize}
        \item 插入新教师需要复制教材信息。
        \item 删除一个教师,可能导致教材信息丢失。
    \end{itemize}
    \item 插入/删除异常。
\end{itemize}

\subsection{4NF 分解要求}

\begin{itemize}
    \item 将关系 $R$ 分解成 $R_1, R_2, \ldots, R_n$,使每个 $R_i$ 满足 4NF。
    \item 分解过程应保持 \textbf{Lossless Join}(无损连接)。
\end{itemize}

\textbf{对子模式 $R_i$ 应保留的依赖}:

\begin{itemize}
    \item $D_i$ 包含以下依赖:
    \begin{itemize}
        \item 所有只涉及 $R_i$ 属性的函数依赖。
        \item 所有形如 $\alpha \twoheadrightarrow (\beta \cap R_i)$ 的多值依赖,且 $\alpha \subseteq R_i$。
    \end{itemize}
\end{itemize}

\subsection{4NF 分解算法}

\begin{enumerate}
    \item 初始 $result := \{R\}$,$done := false$。
    \item 计算 $D^+$。
    \item 设 $D_i$ 为 $D^+$ 限制在 $R_i$ 上的依赖。
    \item \textbf{循环}:
    \begin{itemize}
        \item 若 $result$ 中存在不满足 4NF 的 $R_i$,则执行以下步骤:
        \begin{enumerate}
            \item 取出 $R_i$ 上一个非平凡 MVD $\alpha \twoheadrightarrow \beta$。
            \item 执行分解:
            \[
            result := (result - R_i) \cup (\alpha, \beta) \cup (R_i - \beta)
            \]
        \end{enumerate}
    \end{itemize}
    \item 直到所有 $R_i$ 满足 4NF。
\end{enumerate}

\textbf{保证}:分解后,每个 $R_i$ 在 4NF 中,且分解是 Lossless Join。

\subsection{进一步的范式(Further Normal Forms)}

\begin{itemize}
    \item \textbf{连接依赖}(Join Dependencies, JD)推广了多值依赖。
    \begin{itemize}
        \item 对应 \textbf{第五范式}(5NF),也称 \textbf{投影-连接范式}(PJNF)。
    \end{itemize}
    \item 还有更一般的约束,称为 \textbf{域-码范式}(Domain-Key Normal Form, DKNF)。
\end{itemize}

\textbf{问题}:

\begin{itemize}
    \item DKNF 推理困难。
    \item 没有一套可靠、完备的推理规则。
    \item 实际应用中极少采用。
\end{itemize}

\subsection{小结}

\begin{tabular}{|c|c|}
\hline
\textbf{范式} & \textbf{目标} \\
\hline
BCNF & 消除所有非平凡 FD 依赖导致的数据冗余 \\
4NF & 消除多值依赖(MVD)导致的数据冗余 \\
5NF (PJNF) & 消除连接依赖导致的数据冗余 \\
\hline
\end{tabular}

\begin{itemize}
    \item 4NF 是 BCNF 的增强。
    \item 设计复杂数据库时若存在独立的多值属性,应考虑 4NF 分解。
\end{itemize}
