\section{第一范式}

\subsection{第一范式概念}

定义:如果一个relation中所有属性的domain都是atomic的,则该关系属于第一范式(1NF);

什么是atomic:如果一个域中的元素被认为是不可再分的单元,这个域是atomic。

非原子域的例子:
\begin{itemize}
    \item 复合属性,例如name的属性是name set
    \item 多值属性,例子一个人的多个pjone
    \item 复杂数据类型,面向对象中复杂的对象
\end{itemize}

关系数据库要求:所有关系必须复合1NF,即不能直接存储非原子值。

\subsection{如何处理非原子值}

处理复合属性:
\begin{itemize}
    \item 使用多个属性来表示复合属性中的各组成部分
    \item 例如name $\to$ first name, last-name
\end{itemize}

处理多值属性:
\begin{itemize}
    \item 多字段:person(pname,phone1,phone2,phone3,...) 缺点:字段数目不定,扩展困难。
    \item 单独表:(推荐)phone(pname, phone)每个phone值对应一行,结构灵活、可扩展。
    \item 单字段:person(pname, phones)将多个值拼接到一个字段中,不推荐,违反原子性,难以查询。
\end{itemize}

最合理做法是单独建表来处理多值属性(符合1NF,保持数据库设计规范)

\subsection{非原子策略的缺点}

如果没有正确处理非原子值,会导致以下问题:
\begin{itemize}
    \item 存储复杂:数据结构不一致,难以维护。
    \item 鼓励冗余存储:多字段/拼接字段容易产生冗余。
    \item 查询复杂:SQL 查询变复杂,降低性能。
\end{itemize}

因此,设计中应尽量避免非原子值,确保符合 1NF。

\subsection{原子性是使用方式决定的}

关键观点:原子性不仅是数据的本质,还取决于 使用方式。

示例:
\begin{itemize}
    \item Strings通常被视为不可分
    \item 但如果学号CS0012/EE1127被拆解前两位判断部门$\to$实际上这个字段就不是原子字段了
    \item 这种处理方式会导致在应用程序中编码信息,而不是在数据库中建模,这是不好的设计习惯
\end{itemize}

原因:数据库设计应清晰建模每个语义单元,不要把编码逻辑混入数据字段,避免隐式依赖,提升可维护性。