\section{分解}

\subsection{分解的目标}

分解的目的是:
\begin{itemize}
    \item 判断某个关系$R$是否处于 “好” 的形式(Good form):
       \begin{itemize}
          \item 无冗余存储
          \item 无插入/删除/更新异常
       \end{itemize}
    \item 如果关系$R$不是 Good form,则分解为一组关系:$\{R_1,R_2,...,R_n\}$
\end{itemize}

需要满足以下条件:
\begin{enumerate}
    \item 无损连接分解(Lossless-join decomposition,必须保证)
    \item 依赖保持(Dependency preservation,尽量保证)
    \item 每个子关系$R_i$处于 Good form —— BCNF 或 3NF(优先满足)。
\end{enumerate}

\subsection{分解的良好性质}

\noindent\textbf{1.属性完整性}

分解后,所有原始属性必须包含在子关系中:$R=R_1\cup R_2$

\noindent\textbf{2.无损连接分解}

定义:对任意在$R$上的关系$r$,分解后通过自然连接$\Join$操作应当能恢复原始关系:$r=\pi_{R_1}(r)\Join\pi_{R_2}(r)$

判定条件(适用于两表分解):$R_1\cap R_2\neq\phi$,且必须满足:$(R_1\cap R_2)\to R_1\quad \text{or}\quad (R_1\cap R_2)\to R_2$

直观理解:两个子模式的公共属性如果是某一子关系的候选键,则保证无损连接。保证信息不丢失(lossless),这是分解时的强制性条件,必须满足。

\noindent\textbf{3.依赖保持}

定义:分解后,原有的函数依赖集 F,应尽量能在分解后的子关系$R_i$中直接检验,不需要回到全局 JOIN 后验证。
这样可以避免在更新操作时,必须JOIN才能验证FD约束,提升效率。

具体做法:
\begin{itemize}
    \item 对$F$中的每个依赖$\alpha\to\beta$,希望存在某个$R_i$包含$\alpha\cup\beta$,使得可以直接在$R_i$内部验证。
    \item 理想状态:$(F_1\cup F_2\cup...\cup F_n)^+=F^+$
    \item 注:如果依赖保持不满足,不是非法,但会影响更新代价。
\end{itemize}

\begin{lstlisting}[style=cppstyle]
result := a
while result change:
    for each Ri:
        t := (recult \cap Ri)^+ \cap Ri 
        result := result \cup t
if result contains b, then a\to b is keeping    
\end{lstlisting}

若 F 中所有依赖都通过上述方式被保持,说明该分解是 dependency preserving。

\noindent\textbf{4.无冗余}

希望分解后子关系处于BCNF 或 3NF,进一步减少冗余和异常。

一般来说,BCNF 保证最严格,但有时为了保持依赖,退而求其次用 3NF。