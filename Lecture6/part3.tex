\section{函数依赖}

\subsection{什么是函数依赖}

函数依赖(FD)是一种完整性约束,用于描述关系中属性之间的值的依赖关系。

数学上的类比是$x\to f(x)$,意味着一个属性组$\alpha$的值可以唯一决定另一个属性组$\beta$的值,记作:$\alpha\to\beta$

定义:若对于关系$R$中任意两个元组$t_1,t_2$。若$t_1[\alpha]=t_2[\alpha]$,则$t_1[\beta]=t_2[\beta]$,
则称$\alpha\to\beta$在$R$上成立。

例子:
贷款信息表:F=\{loan-number$\to$amount,loan-number$\to$branch-name,(customer-name,loan-number)$\to$amount,(customer-name,loan-number)$\to$branch-name\}
不希望出现loan-number$\to$cutomer-name 

\subsection{函数依赖与键的关系}

键是函数依赖的一种特殊形式

超键:$K$是$R$的超键$\leftrightarrow K\to R$

候选键:$K\to R$,且$K$没有真子集$\alpha$使得$\alpha\to R$

函数依赖比键更一般,可以表达键无法表达的约束。

\subsection{函数依赖的作用}

1.验证关系是否满足约束:给定关系$r$和函数依赖集$F$,若$r$满足$F$,则称$r$满足$F$

2.规范化设计模式:
\begin{itemize}
    \item 判断模式的规范化程度,帮助消除冗余、提高设计质量。
    \item 可以推导出哪些属性组是否应该归属于同一张表。
\end{itemize}

\subsection{平凡与非平凡依赖}

平凡依赖:$\alpha\to\beta$若$\beta\subseteq \alpha$

非平凡依赖:$\alpha\to\beta$且$\beta\nsubseteq\alpha$

\subsection{函数依赖集的闭包$F^+$}

闭包是指:由$F$推导出的所有可能成立的函数依赖集合。

\subsection{Armstrong公理}

计算闭包的基本推导规则:
\begin{enumerate}
    \item 自反律:$\beta\subseteq\alpha\Longrightarrow\alpha\to\beta$
    \item 增补律:$\alpha\to\beta\Longrightarrow\gamma\alpha\to\gamma\beta$
    \item 传递律:$\alpha\to\beta$且$\beta\to\gamma\Longrightarrow\alpha\to\gamma$
\end{enumerate}

补充规则:
\begin{itemize}
    \item 合并律:$\alpha\to\beta$且$\alpha\to\gamma\Longrightarrow\alpha\to\beta\gamma$
    \item 分解律:$\alpha\to\beta\gamma\Longrightarrow\alpha\to\beta$且$\alpha\to\gamma$
    \item 伪传递律:$\alpha\to\beta$且$\gamma\beta\to\delta\Longrightarrow\alpha\gamma\to\delta$
\end{itemize}

\subsection{属性集闭包$\alpha^+$}

属性集闭包$\alpha^+$是指由$\alpha$函数决定的所有属性集合

用途:
\begin{itemize}
    \item 判断$\alpha$是否是SuperKey(若$\alpha^+=R$)
    \item 判断$\alpha\to\beta$是否在$F^+$中成立(是否$\beta\subseteq\alpha^+$)
\end{itemize}

\subsection{正则覆盖}

目的:压缩函数依赖集,减少检查代价。

特点:最小的等价函数依赖集,去除冗余依赖和无关属性。

三种冗余情况:
\begin{enumerate}
    \item 整个 FD 冗余$\to$可由其他依赖推出。
    \item 左边属性冗余。
    \item 右边属性冗余。
\end{enumerate}

计算过程:反复:
\begin{itemize}
    \item 应用合并律。
    \item 找出无关属性,删除。
    \item 直到 F 不再变化。
\end{itemize}