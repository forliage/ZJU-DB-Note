\section{多值依赖(Multivalued Dependencies)}

\subsection{概述}

\begin{itemize}
    \item 即使在 BCNF 中,有些数据库模式仍然不够规范化。
    \item 示例数据库:\texttt{classes(course, teacher, book)},其中 $(c, t, b) \in$ classes 表示 $t$ 教授课程 $c$,$b$ 是 $c$ 课程所需教材。
    \item 每门课程 $course$ 对应一组教师 $teacher$($1:n$)以及一组教材 $book$($1:n$),教师和教材是独立的多值属性。
    \item \textbf{多值属性}之间相互独立,应该避免交叉组合造成冗余。
\end{itemize}

\subsection{示例数据}

\begin{center}
\begin{tabular}{|l|l|l|}
\hline
\textbf{course} & \textbf{teacher} & \textbf{book} \\
\hline
database & Avi & DB Concepts \\
database & Avi & DB system (Ullman) \\
database & Hank & DB Concepts \\
database & Hank & DB system (Ullman) \\
database & Sudarshan & DB Concepts \\
database & Sudarshan & DB system (Ullman) \\
operating systems & Avi & OS Concepts \\
operating systems & Avi & OS system (Shaw) \\
operating systems & Jim & OS Concepts \\
operating systems & Jim & OS system (Shaw) \\
\hline
\end{tabular}
\end{center}

\begin{itemize}
    \item 目前 \texttt{classes} 关系中只有 trivial FDs,因此该关系在 BCNF 中。
    \item 但存在 \textbf{冗余}和 \textbf{插入异常},例如添加新的教师,需要插入 $course$ 和每本 $book$ 的组合。
\end{itemize}

\subsection{关系分解}

\begin{itemize}
    \item 为避免冗余,推荐将 \texttt{classes} 分解为两个关系:
\end{itemize}

\begin{minipage}{0.45\linewidth}
\centering
\textbf{teaches}
\begin{tabular}{|l|l|}
\hline
\textbf{course} & \textbf{teacher} \\
\hline
database & Avi \\
database & Hank \\
database & Sudarshan \\
operating systems & Avi \\
operating systems & Jim \\
\hline
\end{tabular}
\end{minipage}
\hfill
\begin{minipage}{0.45\linewidth}
\centering
\textbf{text}
\begin{tabular}{|l|l|}
\hline
\textbf{course} & \textbf{book} \\
\hline
database & DB Concepts \\
database & DB system (Ullman) \\
operating systems & OS Concepts \\
operating systems & OS system (Shaw) \\
\hline
\end{tabular}
\end{minipage}

\begin{itemize}
    \item 这样可以保证两个关系分别满足 Fourth Normal Form (4NF)。
\end{itemize}

\subsection{多值依赖的定义}

\begin{itemize}
    \item 设 $R$ 是关系模式,$\alpha, \beta \subseteq R$,若对 $R$ 的任意关系 $r$,所有满足 $t_1[\alpha] = t_2[\alpha]$ 的元组对 $t_1,t_2$,存在 $t_3,t_4$ 满足以下条件:
    \begin{align*}
        t_3[\alpha] &= t_4[\alpha] = t_1[\alpha] = t_2[\alpha] \\
        t_3[\beta] &= t_1[\beta],\quad t_4[\beta] = t_2[\beta] \\
        t_3[R-\alpha-\beta] &= t_2[R-\alpha-\beta] \\
        t_4[R-\alpha-\beta] &= t_1[R-\alpha-\beta]
    \end{align*}
    \item 记作 $\alpha \twoheadrightarrow \beta$,即 $\alpha$ 多值决定 $\beta$。
    \item 若 $\beta \subseteq \alpha$ 或 $\alpha \cup \beta = R$,则 $\alpha \twoheadrightarrow \beta$ 是平凡的。
\end{itemize}

\subsection{多值依赖的理论性质}

\begin{itemize}
    \item \textbf{推导规则}:
    \begin{itemize}
        \item 若 $\alpha \rightarrow \beta$,则 $\alpha \twoheadrightarrow \beta$(即函数依赖是特殊的多值依赖)。
    \end{itemize}
    \item 闭包 $D^+$ 是由所有逻辑推导出的 \textbf{函数依赖} 和 \textbf{多值依赖} 的集合。
    \item 对于简单的多值依赖,可以直接用定义推理。对于复杂依赖,需借助形式推导系统。
\end{itemize}

\subsection{补充示例}

\begin{center}
\begin{tabular}{|l|l|l|}
\hline
\textbf{Employee-name} & \textbf{project} & \textbf{dependent} \\
\hline
smith & p1 & tom \\
smith & p2 & anna \\
smith & p1 & anna \\
smith & p2 & tom \\
julia & p3 & tom \\
julia & p3 & mary \\
julia & p1 & tom \\
julia & p1 & mary \\
\hline
\end{tabular}
\end{center}

\begin{itemize}
    \item Key = \{employee-name, project, dependent\}。
    \item \textbf{多值依赖}的典型场景:employee 对 project 和 dependent 各自具有独立的多值依赖关系。
\end{itemize}

\subsection{总结}

\begin{itemize}
    \item 多值依赖揭示了 BCNF 中可能未充分规范化的场景。
    \item 通过分解,减少冗余和插入/删除异常。
    \item 目标是使关系满足 Fourth Normal Form (4NF)。
\end{itemize}
