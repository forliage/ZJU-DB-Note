\section{第三范式}

\subsection{引入动机}

在某些情况下,BCNF 分解虽然可以消除冗余,但不能保证 \textbf{依赖保持}(Dependency Preservation)。而在实际数据库设计中,更新操作时需要高效地检查函数依赖是否被违反,若丢失依赖保持则必须进行昂贵的 JOIN 操作。

\textbf{解决方案}:引入一个更弱的范式——\textbf{第三范式(3NF)},具备以下特点:

\begin{itemize}
    \item 允许一定的冗余。
    \item 可以在单个子关系中独立检查 FD,\textbf{保证依赖保持}。
    \item \textbf{总是可以}进行无损连接、依赖保持的 3NF 分解。
\end{itemize}

\subsection{定义}

对于关系模式 $R$,若对于 $F^+$ 中的所有函数依赖 $\alpha \to \beta$,至少满足以下任一条件,则 $R$ 满足 \textbf{3NF}:

\begin{enumerate}
    \item $\alpha \to \beta$ 是平凡依赖($\beta \subseteq \alpha$)。
    \item $\alpha$ 是 $R$ 的超码(superkey)。
    \item $\beta - \alpha$ 中的每一个属性 $A$,都是 $R$ 的某个候选键中的主属性。
\end{enumerate}

\textbf{关系}:BCNF $\Rightarrow$ 3NF。3NF 是对 BCNF 的\textbf{最小放宽},用于保证依赖保持。

\subsection{示例}

关系 $R = (J, K, L)$,函数依赖集 $F = \{ JK \to L, L \to K \}$。

\begin{itemize}
    \item 候选键:$JK$、$JL$。
    \item $JK \to L$:$JK$ 是超码,满足。
    \item $L \to K$:$K$ 是候选键中的一部分,满足。
\end{itemize}

因此,$R$ 满足 3NF。但如果分解成 BCNF,$JK \to L$ 不会被保留,导致依赖无法直接检查,需要 JOIN。

\subsection{3NF 中的冗余}

3NF 允许存在冗余,这正是换取依赖保持的代价。

示例 $R(J,K,L)$ 中,$L \to K$ 导致 $(L,K)$ 对重复存储,甚至出现 NULL 值。

\textbf{总结}:

\begin{itemize}
    \item BCNF:彻底消除冗余,但可能丢失依赖保持。
    \item 3NF:允许一定冗余,保证依赖保持。
\end{itemize}

\subsection{3NF 检测方法}

\begin{itemize}
    \item 只需检查 $F$ 中的 FD,不需要检查 $F^+$。
    \item 对每个 $\alpha \to \beta$:
    \begin{itemize}
        \item 计算 $\alpha^+$,判断 $\alpha$ 是否是超码。
        \item 如果不是,检查 $\beta - \alpha$ 中每个属性是否为候选键中的主属性。
    \end{itemize}
\end{itemize}

\textbf{注意}:

\begin{itemize}
    \item 需要找出所有候选键,计算候选键是 NP-hard 问题。
\end{itemize}

\subsection{3NF 分解算法}

\begin{enumerate}
    \item 对 $F$ 求出规范覆盖 $F_c$。
    \item 对 $F_c$ 中每个 $\alpha \to \beta$,若当前已有子关系中无 $\alpha \beta$,则新建子关系 $R_i = (\alpha, \beta)$。
    \item 最后,若当前子关系中无包含任何候选键的关系,补充一个包含候选键的子关系 $R_{key}$。
\end{enumerate}

保证:

\begin{itemize}
    \item Lossless-join。
    \item Dependency-preserving。
\end{itemize}

\subsection{示例}

关系模式:

\begin{quote}
\texttt{Banker-info-schema = (branch-name, customer-name, banker-name, office-number)}
\end{quote}

函数依赖集:

\begin{quote}
$F = \{ banker-name \to branch-name, office-number;\ customer-name,\ branch-name \to banker-name \}$
\end{quote}

候选键:

\begin{quote}
$\{ customer-name, branch-name \}$
\end{quote}

分解结果:

\begin{quote}
\texttt{Banker-office-schema = (banker-name, branch-name, office-number)}\\
\texttt{Banker-schema = (customer-name, branch-name, banker-name)}
\end{quote}

分解后,$Banker-schema$ 包含候选键,分解结束。

\subsection{3NF 与 BCNF 的比较}

\begin{table}[h!]
\centering
\begin{tabular}{|l|l|l|}
\hline
\textbf{方面} & \textbf{BCNF} & \textbf{3NF} \\
\hline
是否 Always Lossless-join & 是 & 是 \\
是否 Always Dependency-preserving & 否 & 是 \\
是否可能有冗余 & 否 & 有可能 \\
\hline
\end{tabular}
\caption{BCNF 与 3NF 的比较}
\end{table}

\subsection{设计目标总结}

设计目标优先级:

\begin{itemize}
    \item BCNF。
    \item Lossless join。
    \item Dependency preservation。
\end{itemize}

如果不能全部达成,通常选择:

\begin{itemize}
    \item 接受 3NF(允许冗余,保证依赖保持)。
    \item 或接受 BCNF(消除冗余,牺牲依赖保持)。
\end{itemize}

\subsection{跨关系检测 FD}

若分解未保持依赖,可用 \textbf{物化视图(Materialized View)} 辅助检查:

\begin{itemize}
    \item 物化视图定义为 $\alpha\beta$ 上的投影。
    \item 视图可定义 $\alpha$ 为视图上的候选键。
    \item 检查 $\alpha \to \beta$ 变成检查 $\alpha$ 是否为视图的候选键,效率更高。
\end{itemize}

\textbf{代价}:

\begin{itemize}
    \item 空间代价:需要存储物化视图。
    \item 时间代价:视图需要及时更新。
    \item 部分数据库系统不支持视图上的候选键声明。
\end{itemize}

