\section{关系数据库设计中的陷阱}

\subsection{概述}

设计目标:找到一组“好”的关系模式(relation schemas),以保证数据的存储和操作的质量。

坏设计(Bad design)会导致:
\begin{itemize}
    \item 冗余存储
    \item 插入/删除/更新异常
    \item 信息丢失或难以表达
\end{itemize}

示例:关系模式 Lending-schema:(branch-name, branch-city, assets, customer-name, loan-number, amount)
该模式中混合了分支信息 + 客户信息 + 贷款信息,设计上存在冗余和异常风险。

\subsection{分解}

主精炼技术(refinement technique):将一个大表(如 ABCD)拆分成多个子表:(AB), (BCD) 或 (ACD), (ABD) 等。

分解的要求:
\begin{itemize}
    \item 所有原始属性必须出现在分解后的表中,即$R=R_1\cup R_2$
    \item 无损连接分解(Lossless-join decomposition):任意$r$(在$R$上的关系),应有:$r=\pi_{R_1}(r)\Join \pi_{R_2}(r)$
\end{itemize}

\subsection{设计目标和理论基础}

目标:
\begin{itemize}
    \item 判断一个关系 R 是否是“好”的形式(Good form):无冗余(No redundant)。
    \item 若 R 不是 good form,需将其分解为:$\{R_1,R_2,...,R_n\}$
       使得:每个子关系都在 good form。
       整体分解是 无损连接分解(Lossless-join decomposition)。
\end{itemize}

理论依据:
\begin{itemize}
    \item 函数依赖:决定属性之间的约束关系。
    \item 多值依赖:处理属性取多值时的依赖关系。
\end{itemize}