\section{Boyce-Codd范式}

\subsection{BCNF定义}

定义:关系模式$R$满足BCNF,针对一个函数依赖集$F$,如果对于$F^+$中的所有函数依赖$\alpha\to\beta$,
都满足以下两个条件之一:
\begin{enumerate}
    \item $\alpha\to\beta$是平凡依赖(即$\beta\subseteq\alpha$)
    \item $\alpha$是$R$的superkey,即$R\subseteq\alpha^+$或$\alpha\to R$
\end{enumerate}

BCNF 更严格,要求所有的非平凡依赖,左边必须是超码,否则视为违反 BCNF。

\subsection{如何测试BCNF}

判断流程:
对于每一个$\alpha\to\beta$:
\begin{enumerate}
    \item 计算$\alpha^+$(属性闭包)
    \item 判断$\alpha^+$是否包含所有$R$中的属性。若是,$\alpha^+$是superkey;否则违反BCNF
\end{enumerate}

简化测试:
\begin{itemize}
    \item 通常只需检查$F$中的依赖是否违反BCNF
    \item 如果$F$中的都合法,则$F^+$中也不会违反BCNF
\end{itemize}

注意:检查 分解后的子关系 时,必须对$F^+$来判断是否满足 BCNF,而不能只看$F$

\subsection{BCNF分解算法}

\begin{lstlisting}[style=cppstyle]
result := {R}
done := false
compute F^+
while (not done) do 
    if exists any Ri not satisfy BCNF:
        choose a non-trivial dependence a\to b:
            a is not superkey
        split Ri as:
            R1 := (a,b)
            R2 := Ri - b
        update result
    else
        done := true   
\end{lstlisting}

最终 result 中所有子模式都是 BCNF,且分解是 无损连接。

\subsection{BCNF 与依赖保持}

BCNF 分解不一定能保持依赖!

例子:$R=(J,K,L),F=\{JK\to L,L\to K\}$。候选键:JK,JL,R不满足BCNF。任意分解都会破坏$JK\to L$的保持性。

Lossless join,BCNF和dependency preservation三者不可兼得。所以实践中有时会使用 3NF 替代 BCNF,因为 3NF 可以保证依赖保持。

