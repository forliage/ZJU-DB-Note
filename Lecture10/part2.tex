\section{查询成本的度量}

成本通常以回答查询的总耗时来衡量。许多因素会导致时间成本:磁盘访问+CPU+网络通信

通常,磁盘访问是主要主要成本,并且相对容易估算。估算时需考虑以下因素:

\begin{itemize}
    \item 执行的寻道操作次数
    \item 读取的块数$\times$平均块读取成本
    \item 写入的块数$\times$平均块写入成本
    \item 注:写入一个块的成本大于读取一个块的成本,数据写入后被读回,以确保写入成功
\end{itemize}

为简单起见,我们仅使用从磁盘进行的块传输次数和寻道次数作为成本度量:

\begin{itemize}
    \item $t_T$:传输一个块的时间($\approx 0.1ms$)
    \item $t_S$:一次寻道的时间($\approx 4ms$)
    \item $b$次块传输加上$S$次寻道的成本
\end{itemize}
$$\Longrightarrow b*t_T+S*t_S$$

为了简单起见,我们忽略CPU成本,但实际系统确实会考虑CPU成本。

我们的成本公式中不包括将最终结果写回磁盘的成本。

成本取决于主存中缓冲区的大小:

\begin{itemize}
    \item 拥有更多内存可减少对磁盘访问的需求
    \item 用于缓冲的实际可用内存量取决于其它并发的操作系统进程,并且在实际执行前很难确定。
    \item 我们通常使用最坏情况估计(假设操作仅能使用所需的最小内存量)和最好情况估计。
    \item 所需数据可能已存在于缓冲区中,从而避免了磁盘I/O。但在成本估算时很难考虑到这一点。
\end{itemize}