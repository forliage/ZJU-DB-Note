\section{连接操作}

几种不同的连接实现算法:
\begin{itemize}
    \item 嵌套循环连接
    \item 块嵌套循环连接
    \item 索引嵌套循环连接
    \item 合并连接
    \item 哈希连接
\end{itemize}

示例使用以下信息:
\begin{itemize}
    \item 选课表的记录数:$n=10,000$,学生表:$n=5000$
    \item 选课表的块数:$b=400$,学生表:$b=100$
\end{itemize}

\subsection{嵌套循环连接}

为了计算$\theta$连接$r\Join_{\theta}s$
\begin{lstlisting}[style=sqlstyle]
for each tuple tr in r do begin
   for each tuple ts in s de begin
      test pair(tr, ts) to see if they satisfy the join condition theta
      if they do, add tr*ts to the result
    end
end    
\end{lstlisting}

$r$称为连接的外关系,$s$称为连接的内关系。

不需要索引,可用于任何类型的连接条件。

开销大,因为它会检查两个关系的每一对元组。

在最坏的情况下,如果内存仅够容纳每个关系的一个块,则估计成本为$n_r*b_s+b_r$ block transfers+$n_r+b_r$ seeks

如果较小的关系能完全放入内存,则将其用作内关系:将成本降低至$b_r+b_s$次块传输和2次寻道

假设在最坏的内存可用情况下,成本估计为:
\begin{itemize}
    \item 以学生表作为外关系:5000*400+100=2,000,100块传输,5000+100=5100查找
    \item 以外层关系作为输入:10000*100+400=1,000,400块传输和10,400次查找
\end{itemize}

如果较小的关系(学生)能完全放入内存,成本估算将为500次块传输。

\subsection{块嵌套循环连接}

是嵌套循环连接的一种变体,其中内关系的每个块都与外关系的每个块配对。
\begin{lstlisting}[style=sqlstyle]
for each block Br of r de begin
   for each block Bs of s de begin
      for each tuple tr in Br do begin
         for each tuple ts in Bs do begin
            Check if (tr, ts) satisfy the join condition
            if they do, add tr*ts to the result
         end
      end
   end
end    
\end{lstlisting}

最坏情况估计:$b_r*b_s+b_r$次块传输$+2*b_r$次寻道。外层关系中的每个块都会读取一次内层关系中的$s$块。

最佳情况:$b_r+b_s$次块传输+2次寻道

对嵌套循环和块嵌套循环算法的改进:
\begin{itemize}
    \item 在块嵌套循环中,使用$M-2$个磁盘块作为外层关系的块处理单元,其中$M=$为以块为单位的内存大小;使用剩余的两个块来缓冲内层关系和输出:成本$=\lceil  b_r/(M-2) \rceil*b_s+b_r$块传输$+2\lceil b_r/(M-2) \rceil$寻道
    \item 如果等值连接属性构成键或内部关系,则在首此匹配时停止内循环
    \item 交替向前和向后扫描内循环,以利用缓冲区中剩余的块(采用LRU),若可用,使用内部关系上的索引
\end{itemize}

\subsection{索引嵌套循环连接}
如果满足以下条件,索引查找可以替代文件扫描:
\begin{itemize}
    \item 连接是等值连接或自然连接,并且
    \item 内部关系的连接属性上有可用的索引---可以仅为计算连接而构建的一个索引。
\end{itemize}

对于外部关系$r$中的每个元组$t_r$,使用索引在$s$中查找与元组$t_r$满足连接条件的元组。

最坏情况:缓冲区仅能容纳$r$的一个页面,并且对于$r$中的每个元组,我们都要在$s$上执行一次索引查找。

连接成本:$b_r(t_T+t_S)+n_r*c$。其中$c$是遍历索引并为一个元组获取所有匹配的$s$元组的成本,或者是$r$。$c$可以估计为使用连接条件对$s$进行单次选择的成本。

如果$r$和$s$的连接属性上都有索引,则使用元组较少的关系作为外部关系。

\subsection{合并连接(排序归并连接)}

在连接属性上对两个关系进行排序(如果尚未在连接属性上排序)。

合并已排序的关系以进行连接:
\begin{itemize}
    \item 连接步骤类似于排序归并算法的合并阶段。
    \item 主要区别在于处理连接属性中的重复值---连接属性上具有相同值的每一对都必须匹配
\end{itemize}

仅可用于等值连接和自然连接。

每个块只需读取一次(假设连接属性的任何给定值的所有元组都能放入内存)。

因此,合并连接的成本为:$b_r+b_s$块传输$+\lceil b_r/b_b \rceil+\lceil b_s/b_b \rceil$寻道(+如果关系未排序,则为排序成本)

混合归并连接(混合归并连接)如果一个关系已排序,而另一个关系在连接属性上有辅助B+树索引。
\begin{itemize}
    \item 将已排序的关系与B+树的叶节点条目合并
    \item 按未排序关系元组的地址对结果进行排序
    \item 按物理地址顺序扫描未排序关系,并与之前的结果合并,用实际元组替换地址(顺序扫描比随机查找更高效)
\end{itemize}

\subsection{哈希连接}

\subsubsection{Hash连接的基本思想}

1.使用场景:

Hash-Join 主要用于 等值连接(Equi-Join) 和 自然连接(Natural Join),也就是连接条件形式为 r.某属性 = s.某属性 的情形。

2.总流程overview:

假设要对关系(表)R 与 S 进行连接,且它们连接条件为相同的属性(例如 R.A = S.A)。Hash-Join 的基本思路可以分为两个阶段:

\begin{itemize}
    \item 分区阶段:
       \begin{enumerate}
          \item 选定一个哈希函数$h$,将$R$中每个元组tr(tuple)按$h(tr[A])$的哈希值划分到$R_0,R_1,...,R_n$共$n+1$个分区中
          \item 同理,将$S$中每个元组ts按相同的哈希函数$h$,把ts[A]映射到$S_0,S_1,...,S_n$
          \item 这样,所有$R$中哈希值为$i$的元组都进入同一个分区$R_i$,所有$S$中哈希值为$i$的元组都进入分区$S_i$。只需要保证:若$R$中某个元组tr和$S$中某个元组ts能够满足$tr[A]=ts[A]$,那么它们必然在同一个哈希分区$i$中。
       \end{enumerate}
    \item 匹配阶段:对每个分区$i=0,...,n$:
       \begin{enumerate}
          \item 将较小的分区---假设是$S_i$(称为"build"输入)---全部加载到内存中,并基于连接属性再建立一个内存哈希表(使用另一种哈希函数$h'$,仅对这一小块数据进行索引)
          \item 然后对$R_i$(称为"probe"输入)中的每个元组tr,依次读取并在内存哈希表中查找所有匹配的ts。只要找到了,就输出$(tr\Join ts)$的连接结果。
       \end{enumerate}
\end{itemize}

这样,$R_i$仅与$S_i$做内部连接运算,无需与其它$S_j(j\neq i)$分区做比较,因为哈希保证了等值连接的所有候选都落在同一分区。

\subsubsection{Hash-Join的详细步骤}

假设$R$有$b_r$个磁盘块,$S$有$b_s$个磁盘块。同时假设可用内存能容纳$M$个数据页

\noindent 1.选择分区数$n$以及哈希函数$h$。

通常,$n=\lceil b_s/M \rceil\times f$,其中$f$是调节因子,通常取1.2.这样保证每个$S_i$分区大概只有$M$个页面左右,可以一次性装入内存。
$R_i$分区不一定要完全装入内存,但$S_i$一定要能放进内存,才能构建哈希表。

\noindent 2.第一遍---分区阶段:对$S$做哈希分区

\begin{lstlisting}[style=sqlstyle]
 // Reserve a memory output buffer for each partition output_buf[i]
 for (each tuple ts in S read) do:
     idx = h(ts[A])
     write ts to output_buf[idx]
     // When output_buf[idx] is one page full, flush (write) it to the partitioned file S_idx on disk and clear the buffer
 end for
 // Flush all buffers that are not yet full and write the remaining tuples to the corresponding S_i file
\end{lstlisting}

\noindent 3.分区阶段:对$R$做哈希分区

\begin{lstlisting}[style=sqlstyle]
for (each tuple tr in R read) do:
    idx = h(tr[A])
    Write tr to output_buf[idx] of R
    // Write to R_idx file on disk when buffer is full
end for
//  Flush remaining buffers to each R_i partition file   
\end{lstlisting}

\noindent 4.都每个分区$i=0,...,n$:执行Build \& probe
\begin{lstlisting}[style=sqlstyle]
 for i in 0...n do:
     // 4.1  Load S_i into memory first to build the hash table
     load the entire S_i partition into memory
     Create in-memory hash index HSi = <value $\to$ list of records> by hashing each tuple in S_i with join attribute A
     
     // 4.2 Probe R_i tuple by tuple.
     for (each tuple tr in R_i read) do
         key = tr[A]
         Find a list of all matching tuples for key in HSi L = HSi[key]
         for (each ts in L) do:
             Output connection results
         end for
    end for 

    // 4.3 Release HSi, go to next partition i+1
end for
\end{lstlisting}

\subsubsection{Hash-Join优缺点}

\noindent 1.优点:

大多数 IO 都是 顺序写+顺序读,而非随机访问,磁盘吞吐量高。

只要分区合理,每个$S_i$足够小都能驻留内存,Probe 阶段查找非常快。

如果 R 与 S 都是一次性顺序扫描分区,再顺序读$R_i$,磁盘寻道次数少。

\noindent 2.缺点:

需要一次完整地对 R、S 做哈希分区,要写出 n 个分区文件、再读回。对于大表,分区和合并会带来开销。

查询 $R\Join S$ 时,要先完成所有分区并写入磁盘,再做 Build \& Probe。适合批量连接,不适合低延迟单次查询。

如果哈希分区不均匀(Skew),导致某些分区$S_i$超大,会出现“分区倾斜”,或者哈希表溢出,需要额外再递归拆分。

\subsubsection{处理溢出(Overflow)的策略}

\noindent 1.可能原因

某个哈希分区 HSi 过大,无法整体装入内存。

原因可能是连接键的分布不均匀(很多元组的 key 落在同一个桶),或者哈希函数选择不当。

\noindent 2.解决方案

\begin{itemize}
    \item 递归分区(Overflow Resolution):对那个超大的分区$S_i$,再用第二个哈希函数$h_2$ 将其细分成更小的子分区。同时对对应的$R_i$也要用同一$h_2$做分区。这样便可保证“最终子分区”能够满足内存大小。
    \item 预先避免(Overflow Avoidance):在最外层分区时,就尽量让 n 取得更大,使得每个分区平均更小。如果预计某些 key 会非常集中,可以先对 S 做一次非常细的分区,然后再合并分区(分层合并)。
    \item 极端场景退化:如果分区之后仍然无法保证某个构建分区在内存里,那么可以对该分区执行 块嵌套循环连接(Block Nested Loop Join),对重数据量进行笨拙但安全的直接连接。
\end{itemize}

\subsubsection{Hash-Join成本估计}

设不需要递归分区(所有$S_i$都能一次性装进 M 页内存),则:

\noindent\textbf{分区阶段}:
 \begin{itemize}
    \item 对$S$做一次顺序读和分区写$\to 2\times b_s$个块传输
    \item 对$R$做一次顺序读和分区写$\to 2\times b_r$个块传输加上为每个分区写空缓存页的开销(约$n_h$块)
    \item 总计:$3\times (b_r+b_s)+4\times n_h$块传输
 \end{itemize}

 \noindent\textbf{匹配阶段I/O}:
 \begin{itemize}
    \item 对$S_i$直到装入内存已在分区阶段完成,分区读无需额外开销;
    \item 对每个分区$R_i$顺序读一次:$b_r$块
    \item 每次分区查找结束后,不需要再写出连接结束到磁盘。
    \item 再加上每读一个分区要一次磁盘寻道,共$n_h$次。
 \end{itemize}

