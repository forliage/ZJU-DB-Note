\section{基本结构}

\begin{lstlisting}[style=sqlstyle]
SELECT A1,A2,...,An
FROM r1,r2,...,rm
WHERE P    
\end{lstlisting}

$A_i$:属性;$r_i$:关系;$P$:谓词

等价于:$\Pi_{A_1,A_2,...,A_n}(\sigma_P(r_1\times r_2\times ...\times r_m))$

注意:SQL不允许在名称中使用-字符,因此在实际实现中使用branch\_name而不是branch-name.

注意:SQL名称不区分大小写,即可以使用大写或小写字母。

SQL允许关系和查询结果中存在重复项。

要强制消除重复项,请在选择后插入关键字distinct.
\begin{lstlisting}[style=sqlstyle]
SELECT distinct branch_name
FROM loan    
\end{lstlisting}

对立关键字all允许重复.
\begin{lstlisting}[style=sqlstyle]
 SELECT all branch_name
 FROM loan   
\end{lstlisting}

默认情况下,允许重复,即all是默认值。

选择子句中的*表示所有属性
\begin{lstlisting}[style=sqlstyle]
SELECT * FROM loan    
\end{lstlisting}

然而,选择子句可以包含涉及运算+,-,*和/的算术表达式,以及对常量或元组属性的操作:
\begin{lstlisting}[style=sqlstyle]
SELECT loan_number, branch_name, amount * 100
FROM loan    
\end{lstlisting}

WHERE子句指定结果必须满足的条件。

在WHERE子句中,可以使用逻辑连接词包括AND、OR和NOT来组合比较结果,同时可以使用BETWEEN比较运算符
来指定范围。
\begin{lstlisting}[style=sqlstyle]
SELECT loan_number
FROM loan
WHERE amount BETWEEN 90000 AND 100000    
\end{lstlisting}

FROM子句列出了查询中涉及的关系,如果在FROM子句中指定了多个关系,则对应于关系代数的笛卡尔积操作。

SQL允许使用as子句重命名关系和属性:old\_name as new\_name

元组变量通过使用as子句在FROM子句中定义。

SQL包含一个用于字符字符串比较的字符串匹配操作符。模式使用以下两个特殊字符描述:
\begin{itemize}
    \item \%:匹配任何子字符串(类似于文件系统中的*)
    \item \_:匹配任何字符(类似于文件系统中的?)
\end{itemize}

注意:可以实现模糊匹配(放置WHERE子句,并且必须与LIKE操作一起使用)
\begin{lstlisting}[style=sqlstyle]
SELECT customer_name
FROM customer
WHERE customer_name LIKE '%Ze%'    
\end{lstlisting}

我们可以指定desc表示降序,asc表示升序,对于每个属性,升序是默认值。
\begin{lstlisting}[style=sqlstyle]
SELECT * FROM customer
ORDER BY customer_city,customer_street desc, customer_name    
\end{lstlisting}

一些关系代数运算符的多重集版本支持重复。

给定多重集关系$r_1$和$r_2$:
\begin{itemize}
    \item $\sigma_\theta(r_1)$:如果在$r_1$中有$c_1$个元组$t_1$的副本,并且$t_1$满足选择$\sigma_\theta$,则在$\sigma_\theta(r_1)$中有$c_1$个$\sigma_\theta(t_1)$的副本 
    \item $\Pi_A(r_1)$:同理
    \item $r_1\times r_2$:同理
\end{itemize}

SQL重复语义:
\begin{lstlisting}[style=sqlstyle]
SELECT A1,A2,...,An 
FROM r1,r2,...,rm 
WHERE P    
\end{lstlisting}

等价于$\Pi_{A_1,A_2,...,A_n}(\sigma_P(r_1\times r_2\times \dots\times r_m))$