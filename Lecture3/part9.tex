\section{数据库修改}

\begin{lstlisting}[style=sqlstyle]
SELETE FROM <table|view>
[WHERE <condition>]    
\end{lstlisting}

\begin{lstlisting}[style=sqlstyle]
INSERT INTO <table|view>[(c1,c2,...)]
VALUES(e1,e2,...)

INSERT INTO <table|view>[(c1,c2,...)]
SELECT e1,e2,\dots
FROM \dots
\end{lstlisting}

\begin{lstlisting}[style=sqlstyle]
UPDATE <table|view>
SET<c1=e1,[c2=e2,\dots]>[WHERE <condition>]    
\end{lstlisting}

建立在单个基本表上的视图,且视图的列对应表的列,称为"行列视图"。

视图是虚表,对其进行的所有操作都将转化为对基表的操作。

在查询操作时,视图与基表没有区别,但对视图的更新操作有严格限制,例如只有行列视图可以更新数据。

大多数SQL实现仅允许对定义在单一关系上且没有聚合的简单视图进行更新。