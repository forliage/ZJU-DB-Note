\section{数据定义语言}
DDL的主要功能包括:
\begin{itemize}
    \item 为每个关系定义模式
    \item 定义与每个属性相关的值域
    \item 定义完整性约束
    \item 定义每个关系在磁盘上的物理存储结构
    \item 定义每个关系要维护的索引
    \item 定义关系的视图
\end{itemize}

\subsection{SQL中的域类型}

\begin{itemize}
    \item char(n):固定长度字符字符串,长度由用户指定
    \item varchar(n):可变长度字符字符串,最大长度由用户指定$n$
    \item int:整数(一个有限的整数子集,依赖于机器)
    \item smallint:小整数(整数域类型的一个依赖于机器的子集)
    \item numeric(p,d):定点数,具有用户指定的$p$位精度,小数点右侧有$d$位
    \item real,double precision:浮点数和双精度浮点数,具有机器相关的精度
    \item float(n):浮点数,用户指定的精度至少为$n$位
    \item 所有域类型都允许空值。声明某属性为非空将禁止该属性的空值
    \item date:日期(4位数字)年份、月份、日期
    \item Time:一天中的时间,以小时、分钟和秒表示
    \item timestamp:日期+时间
\end{itemize}

SQL中有许多函数用于处理各种类型的数据及其类型转换,但各数据库系统中函数的标准化程度不高。

\subsection{创建表}

\begin{lstlisting}[style=sqlstyle]
CREATE TABLE r(A1D1, A2D2,...,AnDn,
               (integrity constraint 1),
               \dots
               (integrity constraint n))    
\end{lstlisting}

r是关系名称;每个Ai是关系r模式中的属性名称;Di是属性Ai域中值的数据类型。

\subsection{创建表中的完整性约束}

非空

主键$(A_1,A_2,...,A_n)$

检查(P),其中P是一个谓词

\subsection{删除和修改表}

删除表命令会从数据库中删除关于被删除关系的所有信息。

\begin{lstlisting}[style=sqlstyle]
DROP TABLE r    
\end{lstlisting}

使用删除命令时请小心。

修改表命令用于向现有关系添加属性:
\begin{lstlisting}[style=sqlstyle]
ALTER TABLE r ADD A D;
ALTER TABLE r ADD (A1D1, ..., AnDn);    
\end{lstlisting}
其中$A$是要添加到关系$r$的属性名称,$D$是$A$的域。

修改表命令也可以用于删除关系的属性:
\begin{lstlisting}[style=sqlstyle]
ALTER TABLE r DROP A    
\end{lstlisting}
其中$A$是关系$r$中属性的名称。

注意,许多数据库不支持删除属性。

修改表命令也可以用于修改关系的属性。

\subsection{创建索引}

\begin{lstlisting}[style=sqlstyle]
 CREATE INDEX <i-name> ON <table-name> (<attribute-list>);
 
 CREATE UNIQUE INDEX <i-name> ON <table-name> (<attribute-list);

 DROP INDEX <i-name>;
\end{lstlisting}