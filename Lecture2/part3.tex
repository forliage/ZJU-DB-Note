\section{附加关系代数操作}

四种基本符号:Set intersection, Natural Join, Division, Assignment

我们定义了额外的操作,这些操作并不增加关系代数的能力,但简化了常见查询。

\subsection{集合交集操作形式化}

符号:$r\cap s$

定义为:$r\cap s=\{t|t\in r\quad \text{and} t\in s\}$

假设:$r$和$s$具有相同的元数;属性兼容。

$r\cap s=r-(r-s)$

\subsection{自然连接操作的形式化}

符号:$r\Join s$

设$r$和$s$分别是模式$R$和$S$上的关系。那么,$r\Join s$是在模式$R\cup S$上获得的关系,
考虑每对元组来自$r$的$t_r$和来自$s$的$t_s$;如果$t_r$和$t_s$在$R\cap S$的每个属性上具有相同的值,
则将元组$t$添加到结果中,其中$t$在$r$上与$t_r$具有相同的值;$t$在$s$上与$t_s$具有相同的值。

注意:
\begin{enumerate}
    \item $r,s$必须含有共同属性(名称和域都对应相同)
    \item 连接两个关系中同名属性值相等的元组
    \item 结果属性是两个属性集的并集,但消去重名属性
\end{enumerate}

\subsection{$\theta$连接操作的形式化}

符号:$r\Join_{\theta}s$。其中$\theta$是模式中属性的谓词。

$\theta$连接:$r\Join_{\theta}s=\sigma_{\theta}(r\Join s)$

\subsection{除法运算}

适合所有包含"for all"的语句。

实际上,它确定了一个集合是否包含另一个集合。

符号:$r\div s$

设$r$和$s$分别是模式$R$和$S$上的关系,$R=(A_1,...,A_m,B_1,...,B_n)$和$S=(B_1,...,B_n)$。那么,
$r\div s$的结果是模式$R-S=(A_1,...,A_m)$上的一个关系
$$r\div s=\{t|t\in \Pi_{R-S}(r)\land \forall u\in S(tu\in r)\}$$

注意$\Pi_{R-S}(r)$包含了$r\div s$的结果,同时,元组$t$的并集与$s$中的所有元组被$r$覆盖(即,
商来自于$\Pi_{R-S}(r)$,并且其元组$t$与$s$所有元组的拼接被$r$覆盖)

属性/特性:设$q=r\div s$则$q$是满足$q\times s\subset r$的最大关系。

基本代数运算的定义:设$r(R)$和$s(S)$为关系,且设$S\subset R$,则$$r\div s=\Pi_{R-S}(r)-\Pi_{R-S}\left(\left(\Pi_{R-S}(r)\times s\right)-\Pi_{R-S,S}(r)\right)$$

\subsection{赋值操作}

$\gets$提供了一种方便的方式来表达复杂查询。

将查询写成一个由以下组成的顺序程序:一系列赋值;后面跟着一个表达式,其值作为查询的结果显式。

赋值必须始终在临时关系变量中进行

在$\gets$右侧的结果被赋值给左侧的关系变量。

可以在后续表达式中使用变量。

