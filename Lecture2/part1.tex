\section{关系数据库的结构}

\subsection{基本结构}

形式上,给定集合$D_1,D_2,...,D_n(D_i=a_{ij}|_{j=1,...,k})$,一个关系$r$是(一组域$D_i$的笛卡尔积)$D_1\times D_2\times ...\times D_n$的一个子集。

因此,一个关系是一个$n-$元组的集合$(a_{1j},a_{2j},...,a_{nj})$,其中每个$a_{ij}\in D_i(i\in[1,n])$

\subsection{属性类型}

每个关系的属性都有一个名称。

每个属性的允许值集合称为该属性的域。

属性值(通常)要求是原子的,即不可分割的(第一范式)。例如,多值属性不是原子的;复合属性值不是原子的。

特殊值null是每个域的成员。null值在许多操作的定义中会引起复杂性。

\subsection{关系的概念}

关系涉及两个概念:关系模式和关系实例。

关系模式描述了关系的结构;关系实例对应于某一时刻关系中数据的快照。

\subsection{关系模式}

假设$A_1,A_2,...,A_n$是属性。

正则表达:$R=(A_1,A_2,...,A_n)$是一个关系模式。

$r(R)$是关系模式$R$上的一个关系。

\subsection{关系实例}

关系的当前值(即关系实例)由一个表格指定。

一个$t$的元素$r$是一个元组,由表格中的一行表示。

设一个元组变量$t$为一个元组,则$t[name]$表示$t$在name属性上的值。

\subsection{关系的属性}

元组的顺序无关紧要(即元组可以以任意方式存储)

关系中没有重复的元组

属性值是原子的。

\subsection{键}

令$K\subset R$。

如果$K$的值足以唯一标识每个可能关系$r(R)$的元组,那么$K$是$R$的超码。

如果$K$是最小超码,则$K$是候选码。

如果$K$是候选码且由用户明确定义,则$K$是主码(主码通常用下划线标记)。

\subsection{外键}

假设存在关系$r$和$s$:$r(\underline{A},B,C),s(\underline{B},D)$,我们可以说关系
$r$中的属性$B$是引用$s$的外键,而$r$是参照关系,$s$是被参照关系。

参照关系中外码的值必须在被参照关系中实际存在,或为null。

主键和外键是综合约束。