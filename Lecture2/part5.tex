\section{数据库的修改}

可以使用以下操作修改数据库的内容:删除、插入、更新。

所有这些操作都使用赋值运算符表示。

\subsection{删除}

删除请求的表达方式与查询类似,不同之处在于选定的元组从数据库中被移除,而不是显式给用户。

它只能删除整个元组;不能删除某些特定属性上的值。

删除在关系代数中表示为:$r\gets r-E$。其中$r$是一个关系,$E$是一个关系代数查询。

\subsection{插入}

要将数据插入关系中,我们可以:指定要插入的元组;编写一个查询,其结果是一组要插入的元组。

在关系代数中,插入表示为:$r\gets r\cup E$。其中$r$是一个关系,$E$是一个关系代数查询。

单个元组的插入通过让$E$成为一个包含一个元组的常量关系来表示。

\subsection{更新}

一种不更改元组中所有值是情况下更改元组的某个值的机制。

使用广义投影算子来完成此任务:$r\gets \Pi_{F_1,F_2,...,F_n}(r)$。其中每个$F_i$要么是$r$的$i$属性,
如果$i$属性未更新;或者如果属性需要更新,$F_i$是一个仅涉及常量和$r$属性的表达式,给出该属性的新值。