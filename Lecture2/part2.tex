\section{基本关系代数操作}

六个基本运算符:select,project,union,set difference, cartesian product, rename

操作符接受一个或两个关系作为输入,并返回一个新的关系作为结果。

\subsection{选择操作关系示例}

Relation 

\[
r=\begin{array}{|c|c|c|c|}
    \hline
    \mathrm{A} & \mathrm{B} & \mathrm{C} & \mathrm{D}\\ \hline
    \alpha & \alpha & 1 & 7 \\ \hline
    \alpha & \beta & 5 & 7 \\ \hline
    \beta & \beta & 12 & 3\\ \hline
    \beta & \beta & 23 & 10\\ \hline
\end{array}
\]

则

\[
\sigma_{\mathrm{A}=\beta \land \mathrm{D}_{>5}}(r)
=
\begin{array}{|c|c|c|c|}
  \hline
  \mathrm{A} & \mathrm{B} & \mathrm{C} & \mathrm{D} \\\hline
  \beta       & \beta       & 23          & 10          \\\hline
\end{array}
\]

请注意,在进行选择操作时,选择条件需要针对同一元组的属性值。

\subsection{选择操作形式化}

符号:$\sigma_p(r)$

$p$被称为选择谓词

定义为:$\sigma_p(r)=\{t\in r\quad \text{and}\quad p(t)\}$
其中$p$是一个由$\land,\lor,\lnot$连接的命题演算中的公式。

每个项是以下之一:<attribute>op<attribute>or<constant>
其中op是以下之一:$=,\neq,>,\geq,<,\leq$

\subsection{投影操作示例}

\[
r=\begin{array}{|c|c|c|}
    \hline
    \mathrm{A} & \mathrm{B} & \mathrm{C} \\ \hline
    \alpha & 10 & 1 \\ \hline
    \alpha & 20 & 1 \\ \hline
    \beta & 30 & 1 \\ \hline
    \beta & 40 & 2 \\ \hline
\end{array}
\]

\[
\Pi_{A,C}(r)=\begin{array}{|c|c|}
 \hline
 \mathrm{A} & \mathrm{C} \\ \hline
 \alpha & 1\\ \hline
 \alpha & 1 \\ \hline
 \beta & 1 \\ \hline
 \beta & 2\\ \hline   
\end{array}\Longrightarrow 
\begin{array}{|c|c|}
    \hline
    \mathrm{A} & \mathrm{C} \\ \hline
    \alpha & 1\\ \hline
    \beta & 1\\ \hline
    \beta & 2\\ \hline
\end{array}
\]

\subsection{投影操作规范化}

符号:$\Pi_{A_1,A_2,...,A_k}(r)$。其中$A_1,...,A_k$是属性名称,$r$是关系名称。

结果定义为通过删除未列出的列而获得的$k$列的关系。

从结果中删除重复行,因为关系是集合。

\subsection{并集操作示例}

\[
r = \begin{array}{|c|c|}
    \hline
    \mathrm{A} & \mathrm{B} \\ \hline
    \alpha & 1\\ \hline
    \alpha & 2\\ \hline
    \beta & 1\\ \hline
\end{array}
\quad 
s = \begin{array}{|c|c|}
    \hline
    \mathrm{A} & \mathrm{B} \\ \hline
    \alpha & 2\\ \hline
    \beta & 3 \\ \hline
\end{array}
\]

\[
r\cup s = \begin{array}{|c|c|}
    \hline
    \mathrm{A} & \mathrm{B} \\ \hline
    \alpha & 1 \\ \hline
    \alpha & 2 \\ \hline
    \beta & 1 \\ \hline
    \beta & 3 \\ \hline
\end{array}
\]

\subsection{并集操作形式化}

符号:$r\cup s$

定义为:$r\cup s= \{t|t\in r\quad \text{or} \quad t\in s\}$

为了使$r\cup s$有效:$r$和$s$必须具有相同的元数(即,相同的数量的属性);属性域必须兼容。

\subsection{差集运算示例}

\[
r=\begin{array}{|c|c|}
    \hline
    \mathrm{A} & \mathrm{B} \\ \hline
    \alpha & 1 \\ \hline
    \alpha & 2 \\ \hline
    \beta & 1\\ \hline
\end{array}
\quad
s=\begin{array}{|c|c|}
    \hline
    \mathrm{A} & \mathrm{B} \\ \hline
    \alpha & 2\\ \hline
    \beta & 3\\ \hline
\end{array}
\]

\[
r-s=\begin{array}{|c|c|}
    \hline
    \mathrm{A} & \mathrm{B}\\ \hline
    \alpha & 1\\ \hline
    \beta & 1 \\ \hline
\end{array}
\]

\subsection{差集运算形式化}

表示法:$r-s$

定义为:$r-s=\{t|t\in r\quad \text{and}\quad t\notin s\}$

差集必须在兼容关系之家进行:$r$和$s$必须具有相同的元数,$r$和$s$的属性域必须兼容。

\subsection{笛卡尔积操作示例}

\[
r=\begin{array}{|c|c|}
    \hline
    \mathrm{A} & \mathrm{B} \\ \hline
    \alpha & 1 \\ \hline
    \beta & 2 \\ \hline
\end{array}
\quad
s=\begin{array}{|c|c|c|}
    \hline
    \mathrm{C} & \mathrm{D} & \mathrm{E} \\ \hline
    \alpha & 10 & a\\ \hline
    \beta & 10 & a\\ \hline
    \beta & 20 & b\\ \hline
    \gamma & 10 & b\\ \hline
\end{array}
\]

\[
r\times s = \begin{array}{|c|c|c|c|c|}
    \hline
    \mathrm{A} & \mathrm{B} & \mathrm{C} & \mathrm{D} & \mathrm{E} \\ \hline
    \alpha & 1 & \alpha & 10 & a \\ \hline
    \alpha & 1 & \beta & 10 & a \\ \hline
    \alpha & 1 & \beta & 20 & b\\ \hline
    \alpha & 1 & \gamma & 10 & b\\ \hline
    \beta & 2 & \alpha & 10 & a\\ \hline
    \beta & 2 & \beta & 10 & a \\ \hline
    \beta & 2 & \beta & 20 & b\\ \hline
    \beta & 2 & \gamma & 10 & b\\ \hline
\end{array}
\]

\subsection{笛卡尔积操作形式化}

表示法:$r\times s$

定义为:$r\times s = \{(t,q)|t\in r\quad \text{and}q\in s \}$

假设$r(R)$和$s(S)$是属性是互不相交的(即$R\cap S=\phi$)。如果$r(R)$和$s(S)$的属性不是互不相交的,则必须对属性进行重命名。

\subsection{重命名操作}

允许我们为关系代数表达式的结果命名,从而引用它们(过程性)。

允许我们用多个名称引用一个关系。例如,$\rho_x(E)$在名称$X$下返回表达式$E$。
如果关系代数表达式$E$的元数为$n$,则$\rho_{X(A_1,A_2,...,A_n)}(E)$(对关系$E$及其属性进行重命名)
返回表达式的结果$E$