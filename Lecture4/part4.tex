\section{嵌入式SQL}

Embedded SQL是把SQL语句嵌入到“宿主语言”中的一种标准机制。

为什么要用?
\begin{itemize}
    \item SQL功能不完备。例如流程控制、复杂计算、资源管理(文件、网络)等,要在宿主语言里处理更方便。
    \item 动态参数传递:通过宿主语言变量,可以在运行时构造查询条件并把结果带回程序
    \item 事务与业务逻辑结合:在同一程序中既能执行业务算法,又能做数据库操作。
\end{itemize}

1.Pre-processing:嵌入式SQL代码需要先由“预编译器”(如 pro*c、esql、nsqlprep 等)扫描,把所有 EXEC SQL ... END\_EXEC 块中的 SQL 提取出来,生成普通函数调用(OCI、ODBC、CLI、LIBPQ 等),并保留宿主语言变量接口。
预编译器还会插入头文件(如 \#include <sqlca.h>)、定义 SQL 通讯区 SQLCA,用于保存每条SQL执行后的状态码(如 sqlca.sqlcode/sqlca.sqlerrd)。

2.DECLARE section
\begin{lstlisting}[style=cppstyle]
    EXEC SQL BEGIN DECLARE SECTION;
    char V_an[20], bn[20];
    float bal;
  EXEC SQL END DECLARE SECTION;
\end{lstlisting}

3.Single-Row Query:
\begin{lstlisting}[style=cppstyle]
    scanf("%s", V_an); 
    EXEC SQL
      SELECT branch_name, balance
        INTO :bn, :bal
        FROM account
       WHERE account_number = :V_an;
    END_EXEC;      
\end{lstlisting}

4.Cursor:
\begin{lstlisting}[style=cppstyle]
    EXEC SQL
      DECLARE c CURSOR FOR
        SELECT B.customer_name, B.customer_city
          FROM depositor D, customer B, account A
         WHERE D.customer_name = B.customer_name
           AND D.account_number = A.account_number
           AND A.balance > :v_amount;
    END_EXEC;
    EXEC SQL OPEN c END_EXEC;
    while (1) {
      EXEC SQL FETCH c INTO :cn, :cc END_EXEC;
      if (sqlca.sqlcode == 100) break; 
      printf("customer:%s, city:%s\n", cn, cc);
    }
    EXEC SQL CLOSE c END_EXEC;
       
\end{lstlisting}

5.单行更新:
\begin{lstlisting}[style=cppstyle]
    scanf("%f", &delta);
    EXEC SQL
      UPDATE account
         SET balance = balance + :delta
       WHERE account_number = :V_an;
    END_EXEC;        
\end{lstlisting}

6.多行更新:
\begin{lstlisting}[style=cppstyle]
    EXEC SQL
    DECLARE csr CURSOR FOR
      SELECT * FROM account
       WHERE branch_name = 'Perryridge'
       FOR UPDATE OF balance; 
  END_EXEC;
  
  EXEC SQL OPEN csr;
  while (1) {
    EXEC SQL FETCH csr INTO :an, :bn, :bal;
    if (sqlca.sqlcode == 100) break;
    EXEC SQL
      UPDATE account
         SET balance = balance + 100
       WHERE CURRENT OF csr;
    END_EXEC;
  }
  EXEC SQL CLOSE csr;     
\end{lstlisting}