\section{授权}

安全性——防止恶意窃取或修改数据的行为。
\begin{itemize}
    \item 数据库系统级别:身份验证和授权机制仅允许特定用户访问
    \item 操作系统级别:操作系统超级用户可以对数据库为所欲为!需要良好的操作系统级别的安全性
    \item 网络级别:必须使用加密来防止:窃听(未经授权读取消息);维装(冒充授权用户或发送据称是...的消息来自授权用户)
    \item 物理层面:入侵者对计算机的物理访问可能会破坏数据
    \item 人员层面
\end{itemize}

数据库部分内容的授权形式:
\begin{itemize}
    \item 读取授权:允许读取数据,但不允许修改数据
    \item 插入授权:允许插入新数据,但不允许修改现有数据
    \item 更新授权:允许修改数据,但不允许删除数据
    \item 删除授权:允许删除数据
\end{itemize}

修改数据库模式的授权形式:
\begin{itemize}
    \item 索引授权:允许创建和删除索引
    \item 资源授权:允许创建新的关系
    \item 修改授权:允许在关系中添加或修改属性
    \item 删除授权:允许删除关系
\end{itemize}

授权与视图:
\begin{itemize}
    \item 可以为用户授予视图的访问权限,而无需授予视图定义中所有使用关系的任何访问权限
    \item 视图隐藏数据的能力既有助于简化系统的使用,又能通过仅允许用户访问其工作所需的数据来增强安全性。
    \item 可以结合使用关系级别的安全性和视图级别的安全性,以精确限制用户对其所需数据得访问
\end{itemize}

视图授权:由于为创建实际关系,创建视图不需要资源授权。视图创建者仅获得那些不会超出其已有权限的额外授权的特权。


授权从一个用户传递到另一个用户可以用授权图来表示。此图的节点的用户,图的根节点是数据库管理员。

要求:授权图中的所有边必须是数据库管理员的某条路径的一部分。

SQL中授予语句用于授予权限:
\begin{lstlisting}[style=sqlstyle]
GRANT <privilege list> ON <table | view>
   TO <user list>    
\end{lstlisting}

<user list>为:用户ID;public,允许所有有效用户拥有授予的权限;一个role

授予视图的权限并不意味着授予底层关系的任何权限。

权限的授予这必须已经拥有指定项的权限。

\begin{itemize}
    \item select:允许对关系进行读访问,或使用视图进行查询的能力
    \item insert:插入元组的能力
    \item update:使用SQL更新语句进行更新的能力
    \item delete:删除元组的能力
    \item references:在创建关系时声明外键的能力
    \item all privilege:用作所有允许权限的简写形式
    \item all
\end{itemize}

带有授予选项:允许被授予权限的用户将该权限传递给其它用户。

Role:通过创建相应的"Role",可以仅指定一次允许某类用户拥有通用权限的角色。

权限可以像授予或撤销用户权限一样授予或撤销角色的权限;角色可以分配给用户,甚至可以分配给其它角色。

撤销语句用于撤销授权:
\begin{lstlisting}[style=sqlstyle]
 REVOKE <privilege list> ON <table | view>
 FROM <user list> [restrict | cascade]   
\end{lstlisting}

从用户处撤销特权会导致其它用户也失去该特权,这被称为撤销级联。我们可以通过指定restrict来防止级联。

<privilege list>可以为all,用于撤销被撤销者可能持有的所有权限。如果<revoke-list>包含public,则除了那些被明确授予该权限的用户外,所有用户都将失去该权限。

如果同一授予者多次向同一用户授予相同的权限,则该用户在撤销操作后可能仍保留该权限。

所有依赖于被撤销权限的权限也将被撤销。

审计跟踪是数据库所有更改的日志,其中包含诸如执行更改的用户以及更改执行时间等信息。用于跟踪错误/欺诈性更改。可以用trigger实现,但许多数据库系统直接提供。

