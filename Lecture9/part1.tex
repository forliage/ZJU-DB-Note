\section{基本概念}

\subsection{基本概念}

为什么我们需要索引? 索引机制用于加快对所需数据的访问速度,例如,图书馆中的作者目录。

搜索键:用于在文件中查找记录的属性或属性集。

索引文件由以下形式的记录(称为索引条目)组成: Search key + pointer

索引文件通常比原始文件小的多。

两种基本的索引类型(index文件中索引记录如何组织?取决于索引类型):

\begin{itemize}
    \item 有序索引:搜索键(索引条目)按排序顺序存储顺序
    \item 哈希索引:搜索键(索引条目)均匀分布,使用“哈希函数”跨“桶”进行操作。
\end{itemize}

\subsection{索引评估指标}

可高效支持的访问类型,例如:

\begin{itemize}
    \item 属性中具有指定值的记录
    \item 或者属性值落在指定值范围内的记录。
    \item 如:哈希索引不适合'Between'查询条件,但有序索引适用
\end{itemize}

访问时间、插入时间、删除时间、空间开销

时间效率和空间效率是衡量索引技术的最主要指标,也是数据库系统组织和管理技术关注的焦点之一。